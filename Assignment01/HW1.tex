%\title{Assignment 01 of Graduate Algorithms}

%%%%%%%%%%%%%%%%%%%%%%%%%%%%%%%%%%%%%%%%%
% Short Sectioned Assignment
% LaTeX Template
% Version 1.0 (5/5/12)
%
% This template has been downloaded from:
% http://www.LaTeXTemplates.com
%
% Original author:
% Frits Wenneker (http://www.howtotex.com)
%
% License:
% CC BY-NC-SA 3.0 (http://creativecommons.org/licenses/by-nc-sa/3.0/)
%
%%%%%%%%%%%%%%%%%%%%%%%%%%%%%%%%%%%%%%%%%

%-----------------------------------------------------------------------------
% PACKAGES AND OTHER DOCUMENT CONFIGURATIONS
%-----------------------------------------------------------------------------

\documentclass[paper=a4, fontsize=11pt]{scrartcl} % A4 paper and 11pt font size

\usepackage[T1]{fontenc} % Use 8-bit encoding that has 256 glyphs
\usepackage{fourier} % Use the Adobe Utopia font for the document - comment this line to return to the LaTeX default
\usepackage[english]{babel} % English language/hyphenation
\usepackage{amsmath,amsfonts,amsthm,amssymb} % Math packages
\usepackage[lined,boxed,commentsnumbered]{algorithm2e}

\usepackage{sectsty} % Allows customizing section commands
\allsectionsfont{\centering \normalfont\scshape} % Make all sections centered, the default font and small caps

%--------------------
% Quote Style
%--------------------

\usepackage{tikz}
\usetikzlibrary{backgrounds}
\makeatletter

\tikzset{%
  fancy quotes/.style={
    text width=\fq@width pt,
    align=justify,
    inner sep=.2em,
    anchor=north west,
    minimum width=\textwidth,
  },
  fancy quotes width/.initial={.8\textwidth},
  fancy quotes marks/.style={
    scale=2,
    text=white,
    inner sep=0pt,
  },
  fancy quotes opening/.style={
    fancy quotes marks,
  },
  fancy quotes closing/.style={
    fancy quotes marks,
  },
  fancy quotes background/.style={
    show background rectangle,
    inner frame xsep=0pt,
    background rectangle/.style={
      fill=gray!25,
      rounded corners,
    },
  }
}

\newenvironment{fancyquotes}[1][]{%
\noindent
\tikzpicture[fancy quotes background]
\node[fancy quotes opening,anchor=north west] (fq@ul) at (0,0) {$``$};
\tikz@scan@one@point\pgfutil@firstofone(fq@ul.east)
\pgfmathsetmacro{\fq@width}{\textwidth - 2*\pgf@x}
\node[fancy quotes,#1] (fq@txt) at (fq@ul.north west) \bgroup}
{\egroup;
\node[overlay,fancy quotes closing,anchor=east] at (fq@txt.south east) {''};
\endtikzpicture}

\makeatother


%--------------------
% Header and Footer
%--------------------

\usepackage{fancyhdr} % Custom headers and footers
\pagestyle{fancyplain} % Makes all pages in the document conform to the custom headers and footers
\fancyhead[L]{\normalfont \normalsize \textsc{CS673: Graduate Algorithms}} % Class
\fancyhead[R]{\normalfont \normalsize \textsc{Wanzhang Sheng}} % Author
\fancyfoot[L]{} % Empty left footer
\fancyfoot[C]{} % Empty center footer
\fancyfoot[R]{\thepage} % Page numbering for right footer
\renewcommand{\headrulewidth}{0pt} % Remove header underlines
\renewcommand{\footrulewidth}{0pt} % Remove footer underlines
\setlength{\headheight}{13.6pt} % Customize the height of the header

\numberwithin{equation}{section} % Number equations within sections (i.e. 1.1, 1.2, 2.1, 2.2 instead of 1, 2, 3, 4)
\numberwithin{figure}{section} % Number figures within sections (i.e. 1.1, 1.2, 2.1, 2.2 instead of 1, 2, 3, 4)
\numberwithin{table}{section} % Number tables within sections (i.e. 1.1, 1.2, 2.1, 2.2 instead of 1, 2, 3, 4)

% \setlength\parindent{0pt} % Removes all indentation from paragraphs - comment this line for an assignment with lots of text
\setlength{\parskip}{\baselineskip}%
\setlength{\parindent}{0pt}%

%-----------------------------------------------------------------------------
% TITLE SECTION
%-----------------------------------------------------------------------------

\newcommand{\horrule}[1]{\rule{\linewidth}{#1}} % Create horizontal rule command with 1 argument of height

\title{
\horrule{0.5pt} \\[0.4cm] % Thin top horizontal rule
\huge Assignment 01 \\ % The assignment title
\horrule{2pt} \\[0.5cm] % Thick bottom horizontal rule
}

\author{Wanzhang Sheng} % Your name

\date{\normalsize\today} % Today's date or a custom date

\begin{document}

\maketitle % Print the title

%-----------------------------------------------------------------------------
% PROBLEM 1
%-----------------------------------------------------------------------------
\section{}

\begin{fancyquotes}
(8 points) Give a $\Theta (n)$ algorithm that takes as input a sorted list of n integers $L$ and an integer $t$, and determines if there are two elements of $L$ whose sum is exactly $t$. You should give both an English description of how your algorithm works and pseudocode.
\end{fancyquotes}

Basicly, we need to try every possible pair to determine if their sum is $t$.
The target answer space is a set of pair $\{(a, b):a < b \wedge a,b\in L\}$.
If we check every pair, the algorithm should be $\Theta (n^2)$.
However, as a sorted list, some of the pairs could skip so that the algorithm could be $\Theta (n)$.

First, we enumerate every $a$ in upper direction.
Considering currently we pick up $a_1$, and we can search for $t-a_1$ in lower direction.
If it doesn't exist, then we can find $b_1$ which is the first $b$ smaller then $t-a_1$.
Then we can pick up next $a$ as $a_2$, for searching if $t-a_2$ is in the list $L$.
Appearently, $a_1 \leq a_2$, so $t-a_1 \geq t-a_2$.
Which means all $b \in \{b:b \geq b_1 and b \in L\}$ don't have to determine again.
When the search of $a$ meet the search of $b$ and still not found the pair, it means the list $L$ doesn't have such a pair.

Pseudocode as following:

\begin{algorithm}[H]
  $b_1$ = \emph{the last index of} $L$\;
  \For{$a_1$ = \emph{the first index of} $L$; $a_1<b_1$; $a_1=a_1+1$}{
    \While{$L[b_1]>t-L[a_1]$}{
      $b_1 = b_1 - 1$\;
    }
    \If{$L[b_1] = t - L[a_1]$}{
      Function returns true and one possible pair is $(L[a_1], L[b_1])$\;
    }
  }
  Function returns false, which means no such of pair\;
  \caption{Determine the existance of the certain pair.}
\end{algorithm}



%-----------------------------------------------------------------------------
% PROBLEM 2
%-----------------------------------------------------------------------------
\section{}

\begin{fancyquotes}
(2 points) Exercise 2.3-5 Which is asymptotically larger, $\lg^*(\lg(n))$ or $\lg(\lg^*(n))$? Explain!
\end{fancyquotes}

The defination of $\lg^*$ is as following:

\begin{displaymath}
  \lg^*(n) = \left\{
  \begin{array}{lr}
  0 & : n\leq 1\\
  1+\lg^*(\lg(n)) & : n>1
  \end{array}
  \right.
\end{displaymath}

So, for $n>1$, we have:

$$\lg^*(\lg(n)) = \lg^*(n)-1$$

We need to determine the relation between $\lg^*(\lg(n))$ and $\lg(\lg^*(n))$, which means:

$$\lg^*(n)-1 \stackrel{?}{=} \lg(\lg^*(n))$$

Let $k=\lg^*(n)$, so:

$$k-1 \stackrel{?}{=} \lg(k)$$

Apperently, when $n$ gets larger, the $k$ gets larger. And $k-1 \in \Theta(n)$ and $\lg(k)\in\Theta(\lg(n))$, so asymptotically we have:

\begin{align}
\begin{split}
k-1 &> \lg(k)\\
\lg^*(n)-1 &> \lg(\lg^*(n))\\
\lg^*(\lg(n)) &> \lg(\lg^*(n))
\end{split}
\end{align}

Which means $\lg^*(\lg n)$ is asymptotically larger than $\lg(\lg^*(n))$.

%-----------------------------------------------------------------------------
% PROBLEM 3
%-----------------------------------------------------------------------------
\section{}

\begin{fancyquotes}
Use the substitution method to prove that the recurrene relation
$T(n) = 16T(\lceil n/4\rceil)+cn$ is in $\Theta (n^2)$, as follows:

\begin{enumerate}
  \item (2 points) First, use the substitution method to show that $T(n)>cn^2$
  \item (2 points) Next, show that a substitution proof that $T(n)<cn^2$ fails
  \item (2 points) Next, show how to subtract off a lower-order term to make the substitution work.
\end{enumerate}

You can assume that $T(0) = T(1) = C$
\end{fancyquotes}

\subsection{}
For every $n\geq 2$, we have $T(k)>ck^2$, $k<n$, then:
\begin{align}
\begin{split}
T(n) &\geq 16(c\lfloor n/4\rfloor^2)+cn\\
&\geq c(16(\frac{n-3}{4})^2+n)\\
&= c(n^2-5n+9)\\
&\geq cn^2
\end{split}
\end{align}
So, $T(n) \in \Omega(n)$

\subsection{}
Then, try another direction:
\begin{align}
\begin{split}
T(n) &\leq 16(c\lfloor n/4\rfloor^2)+cn\\
&\leq c(16(\frac{n}{4})^2+n)\\
&= c(n^2+n)\\
&\nleq cn^2
\end{split}
\end{align}
Failed.

\subsection{}
Assume, for every $n\geq 2$, we have $T(k)>ck^2-c_0k$, $k<n,c_0\geq \frac{c}{16}$
\begin{align}
\begin{split}
T(n) &\leq 16(c\lfloor n/4\rfloor^2-c_0n)+cn\\
&\leq 16(c(\frac{n}{4})^2-c_0n)+cn\\
&= cn^2-16c_0n+cn\\
&\leq cn^2
\end{split}
\end{align}
So, $T(n)\in O(n^2)$. Since $T(n)\in \Omega(n^2)$, $T(n)\in \Theta(n^2)$.

%-----------------------------------------------------------------------------
% PROBLEM 4
%-----------------------------------------------------------------------------
\section{}

\begin{fancyquotes}
Prove the following bounds using the substitution method.
Recall that to prove $T(n) \in \Theta(f(n))$,
you must show that $T(n) \in O(f(n))$ and $T(n) \in \Omega(f(n))$.
You can assume that $T(0) = T(1) = C$

\begin{enumerate}
  \item (4 points) $T(n) = T(n-3) + n \in \Omega(n^2)$
  \item (4 points) $T(n) = T(n/3) + 1 \in \Omega(\lg n)$
  \item (4 points) $T(n) = T(n/3) + n \in \Omega(n)$
\end{enumerate}
\end{fancyquotes}

\subsection{}
Assume that, for every $n\geq 2$, we have $T(k)\geq ck^2+c_0k$, $k<n$.
\begin{align}
\begin{split}
T(n) &= T(n-3)+n\\
&\geq c(n-3)^2+c_0(n-3)+n\\
&= cn^2-6cn+9c+c_0n-3c_0+n\\
&= cn^2+c_0n+((1-6c)n+(9c-3c_0))
\end{split}
\end{align}
When $1-6c=0$ and $9c-c_0\geq 0$, which means $c=\frac{1}{6}$ and $c_0\leq\frac{1}{2}$, $T(n)\geq cn^2+c_0n$.
So, $T(n)\in\Omega(n^2)$.

Then, assume that, for every $n\geq 2$, we have $T(k)\leq ck^2+c_0k$, $k<n$.
\begin{align}
\begin{split}
T(n) &= T(n-3)+n\\
&\leq c(n-3)^2+c_0(n-3)+n\\
&= cn^2-6cn+9c+c_0n-3c_0+n\\
&= cn^2+c_0n+((1-6c)n+(9c-3c_0))
\end{split}
\end{align}
When $1-6c=0$ and $9c-c_0\leq 0$, which means $c=\frac{1}{6}$ and $c_0\geq\frac{1}{2}$, $T(n)\leq cn^2+c_0n$.
So, $T(n)\in O(n^2)$.
Since $T(n)\in\Omega(n^2)$, $T(n)\in\Theta(n^2)$.
% subsection n^2 (end)

\subsection{}

Assume that, for every $n\geq 2$, we have $T(k)\geq c\lg(k)$, $k<n$.

\begin{align}
\begin{split}
T(n) &= T(\frac{n}{3})+1\\
&\geq c\lg(\frac{n}{3})+1\\
&= c\lg(n)-c\lg(3)+1\\
\end{split}
\end{align}

When $1-c\lg(3)\geq 0$, which means $c\leq\frac{1}{\lg(3)}$, $T(n)\geq c\lg(n)$.
So, $T(n)\in\Omega(\lg(n))$.

Then, assume that, for every $n\geq 2$, we have $T(k)\leq c\lg(k)$, $k<n$.

\begin{align}
\begin{split}
T(n) &= T(\frac{n}{3})+1\\
&\leq c\lg(\frac{n}{3})+1\\
&= c\lg(n)-c\lg(3)+1\\
\end{split}
\end{align}

When $1-c\lg(3)\leq 0$, which means $c\geq\frac{1}{\lg(3)}$, $T(n)\leq c\lg(n)$.
So, $T(n)\in O(\lg(n))$.
Since $T(n)\in\Omega(\lg(n))$, $T(n)\in\Theta(\lg(n))$.

% subsection lg_n (end)

\subsection{}

Assume that, for every $n\geq 2$, we have $T(k)\geq ck$, $k<n$.

\begin{align}
\begin{split}
T(n) &= T(\frac{n}{3})+n\\
&\geq \frac{cn}{3}+n
\end{split}
\end{align}

When $\frac{cn}{3}+n\geq cn$, which means $c\leq\frac{3}{2}$, $T(n)\geq cn$.
So, $T(n)\in\Omega(n)$.

Then, assume that, for every $n\geq 2$, we have $T(k)\leq ck$, $k<n$.

\begin{align}
\begin{split}
T(n) &= T(\frac{n}{3})+n\\
&\leq \frac{cn}{3}+n
\end{split}
\end{align}

When $\frac{cn}{3}+n\geq cn$, which means $c\geq\frac{3}{2}$, $T(n)\leq cn$.
So, $T(n)\in O(n)$.
Since $T(n)\in\Omega(n)$, $T(n)\in\Theta(n)$.

% subsection n (end)

%-----------------------------------------------------------------------------

\end{document}
