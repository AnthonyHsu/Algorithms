\documentclass[paper=a4, fontsize=11pt]{scrartcl} % A4 paper and 11pt font size
\usepackage{./../usfassignment}
\settitle{Assignment 4}
\setauthor{Wanzhang Sheng}
\setcourse{CS673: Graduate Algorithms}

\begin{document}

\maketitle % Print the title

% -----------------------------------------------------------------------------
% PROBLEM 1
% -----------------------------------------------------------------------------
\section{}

\begin{fancyquotes}
  (6 points) Consider a sorted list $L$ of $n$ integers, each of which
  must be distinct (that is, the list contains no duplicates). $L$ may
  contain negative elements. Give an algorithm (describe in English,
  and then give pseudo-code) that finds all indicies $i$ such that
  $L[i] = i$. Your algorithm should run in $O(\lg(n) + k)$ time, where
  $k$ is the number of elements $i$ such that $L[i] = i$.
\end{fancyquotes}

Considering a given position $k$ in $L$, if $L[k]>k$, then
$L[k+1]>k+1$ since they are integers and no duplicates. Which means
that there is no elements in the later of $L$ who will $L[t] = t$.
Simillar if $L[k]<k$, in which case we can drop off the lest part of
the list.

Every time when we are given by a range of the list, we check the most
middle element. If it is equal to the position number, then output
it. Otherwise, we can drop off half of the range by the relationship
between the element and the position number.

Here is the pseudo-code:

\begin{algorithm}[H]
  \SetKwProg{Fn}{Function}{ :}{end}
  \Fn{\textsc{search}{(L, min, max)}}{
    \If{min > max}{return []}
    mid = floor((min+max)/2)\;
    \If{L[mid] < mid}{
      return search(L, mid+1, max)\;
    }
    \If{L[mid] > mid} {
      return search(L, min, mid-1)\;
    }
    return [mid] + search(L, min, mid-1) + search(L, mid+1, max)\;
  }
  \caption{Find all $i$ which $L[i]=i$}
\end{algorithm}

\pagebreak

% -----------------------------------------------------------------------------
% PROBLEM 2
% -----------------------------------------------------------------------------
\section{}

\begin{fancyquotes}
  Problem 8-2 Sorting in place in linear time Suppose that we have an
  array of n data records to sort and that the key of each record has
  the value 0 or 1. An algorithm for sorting such a set of records
  might posses some subset of the following thre desirable
  characteristics:

  \begin{enumerate}
  \item The algorithm runs in $O(n)$ time
  \item The algorithm is stable
  \item The algorithm sorts in place, using no more than a constant
    amount of storage space in addition to the original array
  \end{enumerate}

\end{fancyquotes}

\begin{enumerate}
\item
  \begin{fancyquotes}
    (3 points) Give an algorithm that satisfies criteria 1 and 2 above.
  \end{fancyquotes}

  \begin{algorithm}[H]
    \SetKwProg{Fn}{Function}{\,:}{end}
    \Fn{\textsc{sort\_n\_stable}{(A)}}{
      \For{item in A}{
        \tcc{item.key in [0,1]}
        L[item.key].appendto(item)\;
      }
      return L[0] + L[1]\;
    }
    \caption{O(n) and stable sort.}
  \end{algorithm}

\item
  \begin{fancyquotes}
    (3 points) Give an algorithm that satisfies criteria 1 and 3 above.
  \end{fancyquotes}

  \begin{algorithm}[H]
    \SetKwProg{Fn}{Function}{\,:}{end}
    \Fn{\textsc{sort\_n\_inplace}{(A)}}{
      i = index of first element in A\;
      j = index of last elemetn in A\;
      \While{i < j}{
        \While{(i < j) and (A[i].key == 0)}{i = i+1\;}
        \While{(i < j) and (A[j].key == 1)}{j = j-1\;}
        \If{i < j}{
          swap A[i] and A[j]\;
          i = i+1\;
          j = j-1\;
        }
      }
      return A\;
    }
    \caption{O(n) and inplace sort.}
  \end{algorithm}

\item
  \begin{fancyquotes}
    (3 points) Give an algorithm that satisfies criteria 2 and 3 above.
  \end{fancyquotes}

  \begin{algorithm}[H]
    \SetKwProg{Fn}{Function}{}{end}
    \Fn{\textsc{sort\_stable\_inplace}{(A)}}{
      index = 1\;
      \While{index < A.length}{
        \If{A[index].key == 0}{
          temp = A[index]\;
          prev = index - 1\;
          \While{(A[prev] exist) and (A[prev].key == 1)}{
            A[prev+1] = A[prev]\;
            prev -= 1\;
          }
          A[prev+1] = temp\;
        }
        index += 1\;
      }

      return A\;
    }
    \caption{Stable and inplace sort.}
  \end{algorithm}

\item
  \begin{fancyquotes}
    (3 points) Can you use any of your sorting algorithms from parts
    (a) --- (c) above as the sorting method in line 2 of Radix-Sort,
    so that Radix-Sort sorts $n$ records with b-bit keys in time
    $O(bn)$? Explain how or why not
  \end{fancyquotes}

  Radix-Sort require a sub sort algorithm to be stable. Also, as
  required to sort $n$ records with $b$-bit in $O(bn)$, the sub sort
  algorithm should be able to sort $n$ records in $O(n)$. So we can
  choose the algorithm from (a).

  \begin{algorithm}[H]
    \SetKwProg{Fn}{Function}{}{end}
    \Fn{\textsc{radix\_sort}{(A,b,n)}}{
      \tcc{sort from least significant digit to most significant digit}
      \For{bit from b to 1}{
        \For{item in A}{item.key = item.string[bit]}
        sort\_n\_stable(A)\;
      }
      return A\;
    }
    \caption{Radix-Sort with sort\_n\_stable.}
  \end{algorithm}

\item
  \begin{fancyquotes}
    Suppose that the $n$ records have keys in the range $1 \ldots
    k$. Show how to modify counting sort so that it sorts the records
    in place in $O(n + k)$ time. You may use $O(k)$ storage space
    outside the original array. Is your algorithm stable? (Hint: How
    would you do it for $k = 3$?)
  \end{fancyquotes}

  $O(n+k)$ time with $O(n)$ space stable algorithm:

  \begin{algorithm}[H]
    \SetKwProg{Fn}{Function}{\,:}{end}
    \Fn{\textsc{improved\_counting\_sort\_with\_n\_space} (A)}{
      \tcc{item.key in 1\ldots k}
      \For{item in A}{L[item.key] += 1\;}
      \For{bit in 2\ldots k}{L[bit] += L[bit-1]\;}

      \tcc{Extend original records with target position}
      \For{item in A}{
        \tcc{Assume array start with 0}
        item.target\_position = L[item.key] - 1\;
        L[item.key] += 1\;
      }

      \For{item in A}{
        \While{item.index != item.target\_position}{
          swap(item, A[item.target\_position])\;
        }
      }
    }
    \caption{$O(n+k)$ time with $O(n)$ space stable algorithm.}
  \end{algorithm}

  Or $O(n+k)$ time with $O(k)$ space stable algorithm:

  \begin{algorithm}[H]
    \SetKwProg{Fn}{Function}{}{end}
    \Fn{\textsc{improved\_counting\_sort\_with\_k\_space}{(A)}}{
      \tcc{item.key in 1\ldots k}
      \For{item in A}{L[item.key] += 1\;}
      \For{bit in 2\ldots k}{L[bit] += L[bit-1]\;}
      L[k+1] = n+1\;
      \tcc{Now L for the location of the beginning of the final sorted
        list, add k+1 just for simplification. C for current position.}
      C = L\;
      sorted = 0\;

      \While{sorted < A.size}{
        \For{i in 0\ldots k}{
          \While{(C[i] < L[i+1]) and (A[C[i]].key == i)}{
            C[i] += 1\;
            sorted += 1\;
          }
        }
        \For{i in 0\ldots k}{
          \If{C[i] < L[i+1]}{
            index1 = C[i]\;
            index2 = C[A[C[i]].key]\;
            swap(A[index1], A[index2])\;
            C[A[C[i]].key] += 1\;
            sorted += 1\;
            break\;
          }
        }
      }
      return L[0] + L[1] + \ldots + L[k]\;
    }
    \caption{$O(n+k)$ time with $O(k)$ space stable algorithm}
  \end{algorithm}

\end{enumerate}

\pagebreak

% -----------------------------------------------------------------------------
% PROBLEM 3
% -----------------------------------------------------------------------------
\section{}

\begin{fancyquotes}
  (6 points) Let $X[1\ldots n]$ and $Y [1\ldots n]$ be two arrays,
  each containing $n$ numbers, already in sorted order. Give
  $O(\lg(n))$-time algorithm to find the median of all $2n$ elements
  in arrays $X$ and $Y$.
\end{fancyquotes}

\begin{algorithm}[H]
  \SetKwProg{Fn}{Function}{}{end}
  \Fn{\textsc{median}{(A, B, n)}}{
    mid = floor(n/2)\;
    m1 = A[mid+1]\;
    m2 = B[mid+1]\;
    \If{n == 1}{return (m1+m2)/2\;}
    \If{m1 < m2}{
      return median(A[1::mid], B[n-mid+1::n], n-mid)\;
    }{
      return median(A[n-mid+1::n], B[1::mid], n-mid)\;
    }
  }
  \caption{Find median number of two sorted arrays.}
\end{algorithm}

\pagebreak

% -----------------------------------------------------------------------------
% PROBLEM 4
% -----------------------------------------------------------------------------
\section{}

\begin{fancyquotes}
  Special Cases of Select:
  \begin{enumerate}
  \item (4 points) Give an algorithm to find the second smallest
    element in a list of $n$ elements in at most $n +
    \lceil\lg{n}\rceil - 2$ comparisons. Note: Your algorithm can use
    no more than $n + \lceil\lg{n}\rceil - 2$ total comparisons in the
    worst case. Hint: Find the smallest element, too.
  \item (4 points) Give an algorithm to find the $3rd$ smallest
    element in a list, using the smallest number of worst-case
    comparisons. Exactly how many comparisons does your algorithm take
    in the worst case? Hint: Use your solution to part a | you will
    need to find the smallest and second-smallest element as well!
  \end{enumerate}
\end{fancyquotes}

We can compare the elements by pair.
And take the less one as the parent of these two in a tree.
By this we can gererate a compare tree in $\frac{n}{2} + \frac{n}{4} +
\frac{n}{8} +\ldots+ 1 = n-1$ time.
Then the root of the tree is the smallest element in the array.

Since the second smallest element in the array can only be killed by
the smallest element during the comparisions.
We can track the path of the smallest element to find the smallest of
what it killed.
The number of the elements killed by the smallest element is
$\lg(n) - 1$, so the total comparisions will be $n + \lg(n) - 2$.

\begin{algorithm}[H]
  \SetKwProg{Fn}{Function}{}{end}
  \Fn{\textsc{second}{(A)}}{
    offset = 2**(floor(log(A.length)))-1\tcc*{offset of the all parents}
    L = [Inf]*(2n+1)\tcc*{Init as Inf for the whole tree}
    \tcc{Fill in the leaf nodes}
    \For{index from 1 to A.length}{
      L[index+n] = A[index]\;
    }

    \tcc{Build the tree}
    \For{index from n downto 1}{
      L[index] = min(L[index*2], L[index*2+1])\;
    }

    first = L[1]\tcc*{the smallest one}
    second = Inf\tcc*{the second smallest one we don't know yet}
    index = 1\;
    \While{index < 2n+1}{
      \eIf{L[index*2] == first\tcc*{I hope this not count to the comparisions}}{
        \tcc{It came from the left, and killed right}
        second = min(second, L[index*2+1])\;
        index = index*2\;
      }{
        \tcc{It came from the right, and killed left}
        second = min(second, L[index*2])\;
        index = index*2+1\;
      }
    }

    return second\;
  }
  \caption{Find the second smallest number in the given array.}
\end{algorithm}

It's the same when we finding the $3rd$ smallest element.
It will only be killed by the smallest and the second smallest elements.
The worst case will use $n-1+\lg(n)-1+2(\lg(n)-1) = n + 3\lg(n) - 4$ comparisions.


\begin{algorithm}[H]
  \SetKwProg{Fn}{Function}{}{end}
  \Fn{\textsc{killed}{(L, start, standard)}}{
    killer = L[start]\tcc*{killer is the element we need to follow}
    ans = Inf\tcc*{the smallest killed which we don't know yet}
    place = Inf\tcc*{where the ans be killed}
    index = start\;
    \While{index < 2n+1}{
      \eIf{L[index*2] == killer}{
        \tcc{It came from the left, and killed right}
        \If{(L[index*2+1]>standard) and (ans > L[index*2+1])}{
          ans = L[index*2+1]\;
          place = index*2+1\;
        }
        index = index*2\;
      }{
        \tcc{It came from the right, and killed left}
        \If{(L[index*2]>standard) and (ans > L[index*2])}{
          ans = L[index*2]\;
          place = index*2\;
        }
        index = index*2+1\;
      }
    }

    return ans, place\;
  }

  \Fn{\textsc{third}{(A)}}{
    offset = 2**(floor(log(A.length)))-1\tcc*{offset of the all parents}
    L = [Inf]*(2n+1)\tcc*{Init as Inf for the whole tree}
    \tcc{Fill in the leaf nodes}
    \For{index from 1 to A.length}{
      L[index+n] = A[index]\;
    }

    \tcc{Build the tree}
    \For{index from n downto 1}{
      L[index] = min(L[index*2], L[index*2+1])\;
    }

    first = L[1]\;
    second, place = killed(L, 1, first)\;
    third, place = min(killed(L, 1, second), killed(L, place, second))\;

    return third\;
  }
  \caption{Find the third smallest number in the given array.}
\end{algorithm}

\pagebreak

\end{document}
