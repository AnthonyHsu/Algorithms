%\title{Assignment 02 of Graduate Algorithms}

%%%%%%%%%%%%%%%%%%%%%%%%%%%%%%%%%%%%%%%%%
% Short Sectioned Assignment
% LaTeX Template
% Version 1.0 (5/5/12)
%
% This template has been downloaded from:
% http://www.LaTeXTemplates.com
%
% Original author:
% Frits Wenneker (http://www.howtotex.com)
%
% License:
% CC BY-NC-SA 3.0 (http://creativecommons.org/licenses/by-nc-sa/3.0/)
%
%%%%%%%%%%%%%%%%%%%%%%%%%%%%%%%%%%%%%%%%%

%-----------------------------------------------------------------------------
% PACKAGES AND OTHER DOCUMENT CONFIGURATIONS
%-----------------------------------------------------------------------------

\documentclass[paper=a4, fontsize=11pt]{scrartcl} % A4 paper and 11pt font size

\usepackage[T1]{fontenc} % Use 8-bit encoding that has 256 glyphs
\usepackage{fourier} % Use the Adobe Utopia font for the document - comment this line to return to the LaTeX default
\usepackage[english]{babel} % English language/hyphenation
\usepackage{amsmath,amsfonts,amsthm,amssymb} % Math packages
\usepackage[lined,boxed,commentsnumbered]{algorithm2e}

\usepackage{sectsty} % Allows customizing section commands
\allsectionsfont{\centering \normalfont\scshape} % Make all sections centered, the default font and small caps

%--------------------
% Quote Style
%--------------------

\usepackage{tikz}
\usetikzlibrary{backgrounds}
\makeatletter

\tikzset{%
  fancy quotes/.style={
    text width=\fq@width pt,
    align=justify,
    inner sep=.2em,
    anchor=north west,
    minimum width=\textwidth,
  },
  fancy quotes width/.initial={.8\textwidth},
  fancy quotes marks/.style={
    scale=2,
    text=white,
    inner sep=0pt,
  },
  fancy quotes opening/.style={
    fancy quotes marks,
  },
  fancy quotes closing/.style={
    fancy quotes marks,
  },
  fancy quotes background/.style={
    show background rectangle,
    inner frame xsep=0pt,
    background rectangle/.style={
      fill=gray!25,
      rounded corners,
    },
  }
}

\newenvironment{fancyquotes}[1][]{%
\noindent
\tikzpicture[fancy quotes background]
\node[fancy quotes opening,anchor=north west] (fq@ul) at (0,0) {$``$};
\tikz@scan@one@point\pgfutil@firstofone(fq@ul.east)
\pgfmathsetmacro{\fq@width}{\textwidth - 2*\pgf@x}
\node[fancy quotes,#1] (fq@txt) at (fq@ul.north west) \bgroup}
{\egroup;
\node[overlay,fancy quotes closing,anchor=east] at (fq@txt.south east) {''};
\endtikzpicture}

\makeatother


%--------------------
% Header and Footer
%--------------------

\usepackage{fancyhdr} % Custom headers and footers
\pagestyle{fancyplain} % Makes all pages in the document conform to the custom headers and footers
\fancyhead[L]{\normalfont \normalsize \textsc{CS673: Graduate Algorithms}} % Class
\fancyhead[R]{\normalfont \normalsize \textsc{Wanzhang Sheng}} % Author
\fancyfoot[L]{} % Empty left footer
\fancyfoot[C]{} % Empty center footer
\fancyfoot[R]{\thepage} % Page numbering for right footer
\renewcommand{\headrulewidth}{0pt} % Remove header underlines
\renewcommand{\footrulewidth}{0pt} % Remove footer underlines
\setlength{\headheight}{13.6pt} % Customize the height of the header

\numberwithin{equation}{section} % Number equations within sections (i.e. 1.1, 1.2, 2.1, 2.2 instead of 1, 2, 3, 4)
\numberwithin{figure}{section} % Number figures within sections (i.e. 1.1, 1.2, 2.1, 2.2 instead of 1, 2, 3, 4)
\numberwithin{table}{section} % Number tables within sections (i.e. 1.1, 1.2, 2.1, 2.2 instead of 1, 2, 3, 4)

% \setlength\parindent{0pt} % Removes all indentation from paragraphs - comment this line for an assignment with lots of text
\setlength{\parskip}{\baselineskip}%
\setlength{\parindent}{0pt}%

%-----------------------------------------------------------------------------
% TITLE SECTION
%-----------------------------------------------------------------------------

\newcommand{\horrule}[1]{\rule{\linewidth}{#1}} % Create horizontal rule command with 1 argument of height

\title{
\horrule{0.5pt} \\[0.4cm] % Thin top horizontal rule
\huge Assignment 02 \\ % The assignment title
\horrule{2pt} \\[0.5cm] % Thick bottom horizontal rule
}

\author{Wanzhang Sheng} % Your name

\date{\normalsize\today} % Today's date or a custom date

\begin{document}

\maketitle % Print the title

%-----------------------------------------------------------------------------
% PROBLEM 1
%-----------------------------------------------------------------------------
\section{}

\begin{fancyquotes}
  (4 points) Exercise 4.4-8 Use a recursion tree to give an asymptotically tight solution
  to the recurrence $T(n)=T(n-a)+T(a)+Cn$ where $a\geq 1$ and $C>0$ are constants.
  You can assume that $T(C)$ is $\Omega(1)$ for any constant C.
\end{fancyquotes}


\begin{align}
  \begin{split}
    T(n) =& T(n-a)+T(a)+Cn\\
    =& T(n-2a)+T(a)+Cn\\
    =& T(n-3a)+T(a)+Cn\\
    & \vdots\\
    =& T(a_1)+T(a)+Cn
  \end{split}
\end{align}

So since $T(a)\in \Theta(1)$, we have:
$$T(n)=(\lfloor\frac{n}{a}\rfloor + 1)\times (T(a)+Cn) \in \Theta(n^2)$$


%-----------------------------------------------------------------------------
% PROBLEM 2
%-----------------------------------------------------------------------------
\section{}

\begin{fancyquotes}
  (4 points) Exercise 4.4-9 Use a recursion tree to give an asymptotically tight
  solution to the recurrence $T(n) = T(\alpha n) +T((1-\alpha)n) +Cn$ where $\alpha$
  is a constant in the range $0<\alpha<1$, and $C>0$ is also a constant.
\end{fancyquotes}

\begin{align}
  \begin{split}
    T(n) =& Cn\\
    +& C(\alpha n) + C(1-\alpha)n\\
    +& C(\alpha^2 n) + C\alpha(1-\alpha)n + C(1-\alpha)\alpha n + C(1-\alpha)^2n\\
    +& \,\,\vdots\\
    +& \sum_{i=1}^{2^{k-1}}{C\alpha^{2^k-i+1}(1-\alpha)^{i-1}n}\\
  \end{split}
\end{align}

The $k$ level of recursion tree has $Cn$.
And the depth of the recursion tree is between $\log_{\alpha}{n}$ and $\log_{1-\alpha}{n}$.
So we have $T(n)$ is between $-\log_{\alpha}n\times Cn$ and
$-\log_{1-\alpha}n\times Cn$.
So $T(n)\in\Theta(n\log n)$.


%-----------------------------------------------------------------------------
% PROBLEM 3
%-----------------------------------------------------------------------------
\section{}

\begin{fancyquotes}
  (4 points) Give a tight bound for the recurrence
  $T(n) = 3T(\sqrt{n}) + \lg n$
  by making a change of variables. Your answer will be very similar to the
  change of variables in the last part of Section 4.3 of the text.
\end{fancyquotes}

By renaming $m=\lg n$, we obtain:
$$T(2^m) = 3T(2^{m/2}) + m$$

By renaming $S(m) = T(2^m)$, we obtain:
$$S(m) = 3S(m/2) + m$$

So, by changing back from $S(m)$ to $T(n)$, we obtain:
$$T(n) = T(2^m) = S(m) \in\Theta(m^{\lg3}) = \Theta(\lg^{\lg3}n)$$


%-----------------------------------------------------------------------------
% PROBLEM 4
%-----------------------------------------------------------------------------
\section{}

\begin{fancyquotes}
  (4 points) Give find a tight ($\Omega$) bound for each of the following recurrence relations.
  Assume that $T(n)$ is a constant for suffciently small $n$.
  Prove your solution is correct using either the master method or the substitution method.
\end{fancyquotes}

\begin{enumerate}
\item(4 points) $T(n) = T(n/2) + T(n/4) + n$

  Assume $T(n)\in\Theta(n)$.
  So we have $C_1n \leq T(n)\leq C_2n$.
  And we have $T(1)=C_0$.
  So we have:

  \begin{align}
    \begin{split}
      C_1n/2+C_1n/4+n &\leq T(n) \leq C_2n/2+C_2n/4+n \\
      (3/4C_1+1)n &\leq T(n) \leq (3/4C_2+1)n
    \end{split}
  \end{align}

  When $C_1\leq 4$ and $C_2\geq 4$, we have $C_1n\leq T(n)\leq C_2n$.
  So $T(n)\in\Theta(n)$.

\item(4 points) $T(n) = T(n/2) + T(n/4) + T(n/4) + n$

  Assume $T(n)\in\Theta(n\lg{n})$.
  So we have $C_1n\lg{n}\leq T(n)\leq C_2n\lg{n}$.
  And we have $T(1)=C_0$.
  So we have:

  \begin{align}
    \begin{split}
      C_1n/2\lg{n/2}+2C_1n/4\lg{n/4}+n &\leq T(n) \leq
      C_2n/2\lg{n/2}+2C_2n/4\lg{n/4}+n \\
      C_1n\lg{n}-3/2C_1n+n &\leq T(n) \leq C_2n\lg{n}-3/2C_2n+n
    \end{split}
  \end{align}

  When $C_1\leq 2/3$ and $C_2\geq 2/3$, we have $C_1n\lg{n}\leq T(n)\leq C_2n\lg{n}$.
  So $T(n)\in\Theta(n\lg{n})$.

\item(4 points) $T(n) = 2T(n/2) + \lg{n}$

  Since $\lg{n}\in O(n^{\log_b{a}-\epsilon})=O(n^{\log_2{2}-\epsilon})<O(n)$,
  by Master Method first type, we have: $T(n)\in \Theta(n)$

\item(4 points) $T(n) = 4T(n/4) + n$

  Since $n\in \Theta(n^{\log_b{a}})=\Theta(n^{\log_4{4}})=\Theta(n)$,
  by Master Method second type, we have:
  $T(n)\in \Theta(n^{\log_b{a}}\times \lg{n}=\Theta(n\lg{n}))$

\item(4 points) $T(n) = 3T(n/3) + n\lg{n}$

  Since $n\lg{n}\in \Omega(n^{\log_b{a}>\epsilon})=\Omega(n^{\log_3{3}>\epsilon})>\Omega(n)$,
  and $3n/3\lg{n/3}\leq Cn\lg{n}$,
  by Master Method third type, we have: $T(n)\in \Theta(n\lg{n})$

\item(4 points) $T(n) = 3T(n/9) + \sqrt{n}$

  Since $\sqrt{n}\in \Theta(n^{\log_b{a}})=\Theta(n^{\log_9{3}})=\Theta(n^{1/2})$,
  by Master Method second type, we have: $T(n)\in \Theta(\sqrt{n}\lg{n})$

\item(4 points) $T(n) = 2T(n-1) + n^2$


  Assume $T(n)\in\Theta(2^n)$.
  So we have $C_1{2^n}\leq T(n)\leq C_2{2^n}+C_3n^2$.
  And we have $T(1)=C_0$.
  So we have:

  \begin{align}
    \begin{split}
      2C_1{2^{n-1}}+n^2 &\leq T(n) \leq 2C_2{2^{n-1}}+C_3n^2+n^2 \\
      C_1{2^n}+n^2 &\leq T(n) \leq C_2{2^n}+(C_3+1)n^2 \\
      C_1{2^n} &\leq T(n) \leq C_2{2^n}
    \end{split}
  \end{align}

  True when $C_3\leq -1$.
  So $T(n)\in\Theta(2^n)$.

\end{enumerate}


%-----------------------------------------------------------------------------
% PROBLEM 5
%-----------------------------------------------------------------------------
\section{}

\begin{fancyquotes}
  Problem $4-2$ (12 points) Parameter Passing Costs.
  This is problem $4-3$ in the first edition and second editions.
\end{fancyquotes}

\begin{enumerate}
\item For binary search algorithm, the worst case is found the
  target when the search range shrinked to size one.
  The recurrences are $f(1,N), f(1,N/2), f(1,N/4), f(1,N/8), f(1,N/16) \cdots$.
  And the time cost at the worst case is $O(\lg(N))$.
  \begin{enumerate}
  \item When array is passed by pointer, the upper bound is $O(\lg(N))$.
  \item When array is passed by copying, the upper bound is
    $N+\frac{N}{2}+\frac{N}{4}+\frac{N}{8}+\cdots+1 = O(N)$.
  \item When array is passed by subarray, the upper bound is same $O(N)$.
  \end{enumerate}
\item For Merge-Sort algorithm, the recurrences of the worst case are
  $f(1,N)=f(1,N/2)+f(N/2+1,N)=f(1,N/4)+f(N/4+1,N/2)+f(N/2+1,3N/4)+f(3N/4+1,N)\cdots$.
  And the time cost at the worst case is $\Theta(N\lg(N))$.
  \begin{enumerate}
  \item When array is passed by pointer, the upper bound is $1+2+2^2+2^3\cdots+N=O(N)$.
  \item When array is passed by copying, the upper bound is
    $N+2N+4N+8N+\cdots+N^2 = O(N^2)$.
  \item When array is passed by subarray, the upper bound is same
    $N+2\frac{N}{2}+4\frac{N}{4}+8\frac{N}{8}+\cdots+N = O(N\lg(N))$.
  \end{enumerate}
\end{enumerate}

\end{document}
