\documentclass[paper=a4, fontsize=11pt]{scrartcl} % A4 paper and 11pt font size
\usepackage{./../usfassignment}
\settitle{Assignment 5}
\setauthor{Wanzhang Sheng}
\setcourse{CS673: Graduate Algorithms}

\begin{document}

\maketitle % Print the title

% -----------------------------------------------------------------------------
% PROBLEM 1
% -----------------------------------------------------------------------------
\section{}

\begin{fancyquotes}
  (4 points) Does inserting and then immmediately deleting a unique
  element from a Binary Search Tree always leave the same tree? Does
  inserting and then immediatley deleting a unique element from a
  Red/Black tree always leave the same tree?
\end{fancyquotes}

Insert and delete a unique element in BST \textbf{always} leave the
same tree. Since the element is always be inserted as a leaf, delete a
leaf do not effect other nodes in the BST.

Inserting and deleting a unique element in RBT \textbf{may} change the
tree. Since inserting and deleting both may use rotating which is a
action changed other nodes. Consider inserting a node caused rotating
but the element still a leaf, then after deleting the tree will be
different than before.

%\pagebreak

% -----------------------------------------------------------------------------
% PROBLEM 2
% -----------------------------------------------------------------------------
\section{}

\begin{fancyquotes}
  (8 points) Exercise 14.3-6 (MIN-GAP) Show how to extend a red-black
  tree to support the operation MIN-GAP, which gives the magnitude of
  the difference of the two closest numbers in the tree. For example,
  if $Q = \{1, 5, 9, 15, 18, 22\}$ then MIN-GAP($Q$) returns $18-15 = 3$
  since $15$ and $18$ are the two closest numbers in $Q$.
\end{fancyquotes}

Make $Q$ as a red-black tree $A$, then extend every node to have two
more fields of the minimal and maximal element of the subtree.

\begin{algorithm}[H]
  \SetKwProg{Fn}{Function}{}{end}
  \Fn{\textsc{min-gap}{(A, node)}}{
    return min(node.val - node.left.max, node.right.min - node.val,
    min-gap(A, node.left), min-gap(A, node.right))
  }
  \caption{Min-Gap of the given A in $O(n)$.}
\end{algorithm}

When inserting or deleting a node, we can update the fields of the
nodes in $O(\lg{n})$, by node.min = min(node.val, node.left.min) and
node.max = max(node.val, node.right.max). At the same time, let the
min and max fields of the null node be -inf and inf.

When we need to rotate the nodes, we can use the same method to
update the fields in $O(1)$ for each node, so that the total operation
will stay in $O(\lg{n})$.

%\pagebreak

% -----------------------------------------------------------------------------
% PROBLEM 3
% -----------------------------------------------------------------------------
\section{}

\begin{fancyquotes}
  (8 points) Show how to extend a red-black tree (that contains
  floating point numbers) so that you can calculate the average
  element (arithmetic mean) in constant time (while still taking
  $O(\lg{n})$ time for insert, find and delete).
\end{fancyquotes}

Extend two fields for the sum and the size of the subtree, then the
average number of the subtree by $\text{ave} = \text{sum} / \text{size}$.
And at the same time, no effect of find operation.
When inserting and deleting a element, update the nodes on the path
by $\text{sum}_p = \text{sum}_{lc} + \text{sum}_{rc} + \text{val}_p$,
$\text{size}_p = \text{size}_{lc} + \text{size}_{rc} + 1$, which will
let the time cost stays in $O(\lg{n})$.


%\pagebreak


% -----------------------------------------------------------------------------
% PROBLEM 4
% -----------------------------------------------------------------------------
\section{}

\begin{fancyquotes}
  Problem 14-2 Josephus permutations The Josephus problem is defined
  as follows. Suppose that $n$ people are arranged in a circle and
  that we are given a positive number $m\leq n$. Beginning with a
  designated first person, we proceed around the circle, removing
  every $m$th person. After each person is removed, counting continues
  around the circle that remains. This process continues until all $n$
  people have been removed. The order in which the peope are removed
  from the circle defines the $(n,m)$-Josephus permitation of the
  integers $1,2,\ldots,n$. For example, the $(7,3)$-Josephus
  permutation is $<3,6,2,7,5,1,4>$.
\end{fancyquotes}

\begin{enumerate}
\item
  \begin{fancyquotes}
    (5 points) Suppose that $m$ is a constant. Describe an $O(n)$-time
    algorithm that, given an integer $n$, outputs the $(n,m)$-Josephus
    permutation.
  \end{fancyquotes}

  Just simulate the game process, we can find the number for each time
  in $O(m)$ and delete it in $O(1)$ if we use link-list. Since we need
  find $n$ times, the algorithm will run in $O(mn)$ time. If we treat
  $m$ as a constant, then the algorithm is $O(n)$.

\item
  \begin{fancyquotes}
    (5 points) Suppose that $m$ is not a constant. Describe an
    $O(n\lg{n})$-time algorithm that, given an integer $n$, outputs
    the $(n,m)$-Josephus permutation. Note that by not a constant
    here we mean that $m$ could be a function of $n$, like $n=2$,
    instead of a constant like $5$. $m$ still does not change over the
    course of the algorithm.
  \end{fancyquotes}

  We create a balanced tree for these $n$ numbers in $O(n\lg{n})$ and
  extend the nodes with a additional field indicated the size of the
  subtree in $O(n)$. So that we can determine the rank in the subtree
  of any node in $O(1)$ by $\text{rank} = \text{left-child}.\text{size}+1$.

  Finding the next node by determine the rank $\text{next-rank} =
  ((\text{current-rank} + m -2) \bmod \text{current-size}) + 1$, and
  then find the node in $O(\lg{n})$. Using the equation $(a-1)\bmod n
  +1$ to change the result from $0\dots n-1$ to $1\dots n$. Since we
  need doing $n$ times, the final time cost will be $O(n\lg{n})$.

  Here is the math form:

  \begin{equation*}
    \text{rank}(i)=
    \begin{cases}
      1 & \text{if}\ i=0\ \text{or}\ i=n \\
      ((\text{rank}(i-1)+m-2) \bmod (n-(i-1))) +1 & \text{otherwise}
    \end{cases}
  \end{equation*}

\end{enumerate}


%\pagebreak

\end{document}
