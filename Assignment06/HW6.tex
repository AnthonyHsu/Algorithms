\documentclass[paper=a4, fontsize=11pt]{scrartcl} % A4 paper and 11pt font size
\usepackage{./../usfassignment}
\usepackage{soul}               %Delete line
\settitle{Assignment 6}
\setauthor{Wanzhang Sheng}
\setcourse{CS673: Graduate Algorithms}

\begin{document}

\maketitle % Print the title

% -----------------------------------------------------------------------------
% PROBLEM 1
% -----------------------------------------------------------------------------
\section{}

\begin{fancyquotes}
  (8 ponts) Exercise 15.4--5 Give an $O(n^2)$ algorithm to find the
  longest monotonically increasing subsequence of a sequence of $n$
  numbers.
\end{fancyquotes}

Table: $f(i)$ indicated the longest monotonically increasing
subsequence length with first $i$th number and the $i$th number is
picked.

Formula:

\begin{equation*}
  f(i)=
  \begin{cases}
    1 & \text{if}\ i==1\ \text{or}\ \text{All}\ A[j]>A[j] \\
    \max\{f(j)\}+1 & j<i, A[j]\leq A[i]\ (\text{or}\ A[j]<A[i]\ \text{for
    strictly increasing})
  \end{cases}
\end{equation*}

Order: from $1$ to $n$.

Code:

\begin{algorithm}[H]
  \SetKwProg{Fn}{Function}{}{end}
  \Fn{\textsc{LMIS}{(A, n)}}{
    L = array of n and filled by 0\;
    L[1] = 1\;
    \For{i = 2 $\rightarrow$ n}{
      \For{j = 1 $\rightarrow$ i-1}{
        \tcc{A[j]<A[i] if strictly increasing}
        \If{(A[j] $\leq$ A[i]) and (L[j]>L[i])}
        {
          L[i] = L[j]\;
        }
      }
      L[i] += 1\;
    }
    return L[n]\;
  }
  \caption{The longest monotonically increasing subsequence.}
\end{algorithm}

\pagebreak

% -----------------------------------------------------------------------------
% PROBLEM 2
% -----------------------------------------------------------------------------
\section{}

\begin{fancyquotes}
  (8 points) Problem 15--2 Longest Palendrome subsequence.
\end{fancyquotes}

Table: $f(i,j), 1\leq i\leq j\leq n$ indicated the longest palendrome
subsequence within $i$th number to $j$th number.

Formula:

\begin{equation*}
  f(i,j)=
  \begin{cases}
    0 & \text{if}\ i>j\\
    1 & \text{if}\ i==j\\
    f(i+1,j-1)+2 & \text{if}\ A[i]==A[j]\\
    \max\{f(i+1,j), f(i,j-1)\} & \text{otherwise}
  \end{cases}
\end{equation*}

Order: from smaller range (which means $j-i+1$) to larger range. From
$1$ to $n$.

Code:

\begin{algorithm}[H]
  \SetKwProg{Fn}{Function}{}{end}
  \Fn{\textsc{LPS}{(A, n)}}{
    L = array of n*n filled by 0\;
    \For{i = 1 $\rightarrow$ n}{
      L[i,i] = 1\;
    }

    \tcc{Range from 2 to n}
    \For{k = 1 $\rightarrow$ n-1}{
      \For{i = 1 $\rightarrow$ n-k}{
        j = i+k\;
        \eIf{A[i] == A[j]}{
          L[i,j] = L[i+1,j-1] + 2\;
        }{
          L[i,j] = $\max$(L[i+1,j], L[i,j-1])\;
        }
      }
    }

    return L[1,n]\;
  }
  \caption{Find the longest palendrome subsequence.}
\end{algorithm}

\pagebreak

% -----------------------------------------------------------------------------
% PROBLEM 3
% -----------------------------------------------------------------------------
\section{}

\begin{fancyquotes}
  Consider the alphabet $\Sigma = \{a,b,c\}$ the elements of $\Sigma$
  have the following multiplication table, which is neither
  communative nor associative:

  \begin{tabular}[c]{|c|c|c|c|}
    \hline
    & a & b & c \\
    \hline
    a & b & b & a \\
    \hline
    b & c & b & a \\
    \hline
    c & a & c & c \\
    \hline
  \end{tabular}

  so, $ab = b$, $ba = c$, $bc = a$, $cb = b$, and so on.
\end{fancyquotes}


\begin{enumerate}
\item
  \begin{fancyquotes}
    (8 points) Find a dynamic programming algorithm that examines a
    string $x = x_1x_2x_3\ldots x_n$ and decides whether or not it is
    possible to parenthesize $x$ such that the value of the resulting
    expression is $a$. For example, if $x = bbbba$, your algorithm
    should retun \textsc{yes}, since $(b(bb))(ba) = a$. For the string $x =
    bac$, your algorithm should return \textsc{no} (since $(ba)c = c$ and
    $b(ac) = c$.
  \end{fancyquotes}

  Tables:
  $f(i,j,k), 1\leq i\leq j\leq n, k\in \Sigma$ indicated if it's
  possible to generate $k$ within $i$th to $j$th symbols of $x$.
  $T(m,n)$ is the multiplication table.

  Formula:
  \begin{equation*}
    f(i,j,k)=
    \begin{cases}
      \text{true} & \text{if}\ i==j\ \text{and}\ x[i]==k\\
      \bigcup\{f(i,l,m) \cap f(l+1,j,n) \cap (T[m, n] == k)\} &
      i\leq l<j, m\in\Sigma ,n\in\Sigma
    \end{cases}
  \end{equation*}

  Order: from smaller range (which means $j-i+1$) to larger range.

  Code:

  \begin{algorithm}[H]
    \SetKwProg{Fn}{Function}{}{end}
    \Fn{\textsc{wired\_string\_to\_a}{(x)}}{
      F = array size of n*n*3 filled by false\;

      \For{i = 1 $\rightarrow$ n}{
        F[i,i,x[i]] = true\;
      }

      \For{r = 1 $\rightarrow$ n-1}{
        \For{i = 1 $\rightarrow$ n-r}{
          j = i+r\;
          \For{k $\in\Sigma$}{
            \tcc{Now we have i,j,k. Splitting next.}
            \For{l = i $\rightarrow$ j-1}{
              \For{m $\in\Sigma$}{
                \For {n $\in\Sigma$}{
                  \If{F[i,l,m] and F[l+1,j,n] and (T[m,n]==k)}{
                    F[i,j,k] = true\;
                    next k\;
                  }
                }
              }
            }
          }
        }
      }

      \eIf{F[1,n,a]}{
        return yes\;
      }{
        return no\;
      }
    }
    \caption{Determine whether wired string x can generate a.}
  \end{algorithm}


\item
  \begin{fancyquotes}
    (6 points) Modify your algoritm from part $a$ so that instead of
    returning \textsc{yes} or \textsc{no}, it returns the number of the
    expression can be parenthesized to get the answer $a$.
  \end{fancyquotes}

  \begin{algorithm}[H]
    \SetKwProg{Fn}{Function}{}{end}
    \Fn{\textsc{wired\_string\_to\_a}{(x)}}{
      F = array of n*n*3 filled by false\;
      \textbf{P = array of n*n*3 filled by 0}\;

      \For{i = 1 $\rightarrow$ n}{
        F[i,i,x[i]] = true\;
        P[i,i,x[i]] = 1\;
      }

      \For{r = 1 $\rightarrow$ n-1}{
        \For{i = 1 $\rightarrow$ n-r}{
          j = i+r\;
          \For{k $\in\Sigma$}{
            \tcc{Now we have i,j,k. Splitting next.}
            \For{l = i $\rightarrow$ j-1}{
              \For{m $\in\Sigma$}{
                \For {n $\in\Sigma$}{
                  \If{F[i,l,m] and F[l+1,j,n] and (T[m,n]==k)}{
                    F[i,j,k] = true\;
                    \textbf{P[i,j,k] += P[i,l,m] * P[l+1,j,n]}\;
                    \st{next k}\;
                  }
                }
              }
            }
          }
        }
      }

      \eIf{F[1,n,a]}{
        \st{return yes}\;
      }{
        \st{return no}\;
      }
      return P[1,n,a]
    }
    \caption{How many ways can wired string x generate a.}
  \end{algorithm}

\end{enumerate}
\pagebreak

\end{document}
%%% Local Variables:
%%% mode: latex
%%% TeX-master: t
%%% End:
