\documentclass[paper=a4, fontsize=11pt]{scrartcl} % A4 paper and 11pt font size
\usepackage{./../usfassignment}
\settitle{Assignment 8}
\setauthor{Wanzhang Sheng}
\setcourse{CS673: Graduate Algorithms}

\begin{document}

\maketitle % Print the title

% -----------------------------------------------------------------------------
% PROBLEM 1
% -----------------------------------------------------------------------------
\section{}

\begin{fancyquotes}
  (6 points) Suppose we perform a sequence of $n$ operations on a data
  structure in which the $i$th operation cost is $i$ if $i$ is an
  exact power of 2 and 1 otherwise.
\end{fancyquotes}

\begin{enumerate}
\item
  \begin{fancyquotes}
    Use aggregate analysis to determine the amortized cost per operation.
  \end{fancyquotes}

  \begin{equation}
    \begin{split}
      \sum_{i=1}^{\lg{n}}{(2^{i}-1)} + n\\
      \sum_{i=1}^{\lg{n}}{2^{i}} - \lg{n} + n\\
      2n-1 + n -\lg{n}\\
      3n - \lg{n} -1
    \end{split}
  \end{equation}

  So the amortized cost per operation is $O(1)$.

\item
  \begin{fancyquotes}
    Use the accounting method to determine the amortized cost per operation.
  \end{fancyquotes}

  Charge for all operation cost of $3$.

  \begin{table}[H]
    \centering
    \begin{tabular}[hp]{ccc}
      & actual cost & amortized cost\\
      \hline
      $i = 2^k$ & i & 3\\
      otherwise & 1 & 3
    \end{tabular}
  \end{table}

  For the operation which the cost is $i = 2^k$, it's already paid by
  the $2^{k-1}+1$ to $2^{k}-1$ of $(2^{k-1}-1)*(3-1)=2^k-2$. Plus the
  $3$ paid by $i$, it has $1$ more after $i$th operation.

  So the amortized cost per operation is $O(3) = O(1)$.

\item
  \begin{fancyquotes}
    Use the potential method to determine the amortized cost per operation.
  \end{fancyquotes}

  Let the potential be the $2\times(i - 2^{\lfloor \lg{i}\rfloor})$.

  For the operation which the cost is $1$, the actual cost is $1$ and
  the change in potential is $2$. So the amortized cost is $3$.

  For the operation which the cost is $i$, the actual cost is $i$ and
  the change in potential is:

  \begin{equation}
    \begin{split}
      2\times(i - 2^{\lfloor \lg{i}\rfloor}) - 2\times((i-1) -
      2^{\lfloor \lg{i-1}\rfloor})\\
      2\times(i - 2^{\lg{i}}) - 2\times((i-1) - 2^{\lg{i}-1})\\
      2\times(i - i) - 2\times((i-1) - i/2)\\
      -2\times(i/2-1)\\
      2-i\\
    \end{split}
  \end{equation}

  So the amortized cost is $2$.

  So the amortized for all $i$ operation is $O(1)$.
\end{enumerate}

\pagebreak

% -----------------------------------------------------------------------------
% PROBLEM 2
% -----------------------------------------------------------------------------
\section{}

\begin{fancyquotes}
 (8 points) Design a data strudture that implements the following
 three operations, with the following amortized running times:

 \begin{itemize}
 \item Insert(S,x) inserts element $x$ into a set of
   elements. Required amortized runningtime: O(1)
 \item DeleteLargerHalf(S) deletes the largest $\lceil
   \frac{n}{2}\rceil$ elements from the set. Required amortized
   running time:O(1).
 \item Print(S) Prints out all elements in the set (in any
   order). Required amortized running time:O(n).
 \end{itemize}

 Use the potential method to analyze the running time for each of your operations.
\end{fancyquotes}

Use a array as the data structure.

When Insert a element, just simply add it to the end of the array with
actually time cost $O(1)$.

When DeleteLargerHalf the elements, find the median element in $O(n)$.
Then scan the whole array and insert the element less then median to a
new array in $O(n)$.
Finally delete the old array in $O(n)$.
So the actually cost for DeleteLargerHalf is $O(n)$.

When Print all the elements, just print the array.

Let the potential function is $g(n)=2n$.
We can get the table as \ref{tab:2}:

\begin{table}[hp]
  \centering
  \begin{tabular}[hp]{ccccc}
    & Actually cost & Potential change & amortized cost & Running time\\
    \hline
    Insert & 1 & 2 & 3 & $O(1)$\\
    DeleteLargerHalf & n & -n & 0 & $O(1)$\\
  \end{tabular}
  \caption{Amotized time analysis.}
  \label{tab:2}
\end{table}

So the amortized time cost for Insert and DeleteLargerHalf is $O(1)$
and $O(1)$.

\pagebreak

% -----------------------------------------------------------------------------
% PROBLEM 3
% -----------------------------------------------------------------------------
\section{}

\begin{fancyquotes}
  (6 points) Consider a data structure that consists of a standard
  sorted array, which takes time $O(n)$ to insert into and delete
  from. Create a potential function for this data structure that
  results in amortized time $O(n)$ for insertion and $O(1)$ for
  deletion. Prove the amortized running times. Note that you are not
  supposed to change the operations of insert and delete, just compute
  amortized running times.
\end{fancyquotes}

Let $n$ be the number of the elements in the data structure.
Let the potential be $g(n) = \frac{n^2}{2}$.
The potential will always be greater than $0$.

\begin{table}[hp]
  \centering
  \begin{tabular}[hp]{ccccc}
    & Actual Cost & Potential Change & Amortized Cost & Running Time\\
    \hline
    Insert & $n$ & $n-\frac{1}{2}$  & $2n-\frac{1}{2}$ & $O(n)$\\
    Delete & $n$ & $-n+\frac{1}{2}$ & $\frac{1}{2}$   & $O(1)$\\
  \end{tabular}
\end{table}

\pagebreak

% -----------------------------------------------------------------------------
% PROBLEM 4
% -----------------------------------------------------------------------------
\section{}

\begin{fancyquotes}
  (8 points) Problem 17-2 Making Binary Search Dynamic

  Binary search of a sorted array takes logarithmic search time, but
  the time to insert a new element is linear in the size of the
  array. We can improve the time for insertion by keeping several
  sorted arrays.

  Specifically, suppose that we wish to support \textsc{SEARCH} and
  \textsc{INSERT} on a set of $n$ elements. Let $k =
  \lceil\lg{n}\rceil$, and let the binary representation of $n$ be
  $<n_{k-1},n_{k-2},\ldots,n_0>$. We have $k$ sorted arrays $A_0,
  A_1,\ldots, A_{k-1}$, where for $i = 0,1,\ldots,k-1$, the length of
  array $A_i$ is $2^i$. Each array is either full or empty, depending
  on whether $n_i = 1$ or $n_i = 0$, respectively. The total number of
  elements held in all $k$ arrays is therefore $\sum_{i=0}^{k-1}{n_i
    2^i} = n$. Although each individual array is sorted, elements in
  different arrays bear no particular relationship to each other.
\end{fancyquotes}

\begin{enumerate}
\item
  \begin{fancyquotes}
    Describe how to perform the \textsc{SEARCH} operation for this data
    structure. Analyze its worst-case running time.
  \end{fancyquotes}

  Enumerate $k$ array, for each array use old binary search algorithm
  to perform the search in $O(\lg{2^k})=O(k)$ until find. Return not found
  if all arrays are searched.

  The worst running time:

  \begin{equation}
    \begin{split}
      & \sum_{k=1}^{\lg{n}}{k}\\
      =& \frac{(1+\lg{n})\times\lg{n}}{2}\\
      =& \frac{\lg^2{n}}{2} + \frac{\lg{n}}{2}
    \end{split}
  \end{equation}

  So the worst-case running time is $O(\lg^2{n})$.

\item
  \begin{fancyquotes}
    Describe how to perform the \textsc{INSERT} operation. Analyze its
    worst-case and amortized running times.
  \end{fancyquotes}

  Create a new array of size $1$ contains the element to insert.
  Repeat merge two arrays which have the same size $t$ to a new array
  of size $2t$ by standard merge sort in time cost $2t$, until no
  arrays to merge.

  In the worst case, all the array will be merged once, so the wost
  time cost is $O(t^2)=O(\lg^2{n})$.

  Consider the binary representation of $n$.
  Let $j$ be the first place which $n_j$ is $0$.
  Then the time cost is $\sum_{i=0}^{j-1}{2\times 2^i} =
  \sum_{i=1}^{j}{2^i} = 2^{j+1}-2$.
  For $n$ operations time cost is
  $\sum_{i=1}^{\lg{n}}{\frac{n}{2^i}\times (2^{i+1}-2)} =
  2n(\lg{n}-1+\frac{1}{n} = O(n\lg{n})$.

  So the amortized time cost is $O(\lg{n})$.

\item
  \begin{fancyquotes}
    Discuss how to implement \textsc{DELETE}.
  \end{fancyquotes}

  Search the element to delete in $O(\lg^2{n})$.
  Assume it's in array $A_i$.
  Break the array without the element to delete into $i-1$ arrays
  which the sizes are $2^{i-1}, 2^{i-2}, \ldots, 1$.
  Merge the arrays with the same size until no arrays to merge.

\end{enumerate}

\pagebreak

\end{document}
