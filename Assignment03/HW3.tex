% \title{Assignment 03 of Graduate Algorithms}

%%%%%%%%%%%%%%%%%%%%%%%%%%%%%%%%%%%%%%%%%
% Short Sectioned Assignment
% LaTeX Template
% Version 1.0 (5/5/12)
%
% This template has been downloaded from:
% http://www.LaTeXTemplates.com
%
% Original author:
% Frits Wenneker (http://www.howtotex.com)
%
% License:
% CC BY-NC-SA 3.0 (http://creativecommons.org/licenses/by-nc-sa/3.0/)
%
%%%%%%%%%%%%%%%%%%%%%%%%%%%%%%%%%%%%%%%%%

% -----------------------------------------------------------------------------
% PACKAGES AND OTHER DOCUMENT CONFIGURATIONS
% -----------------------------------------------------------------------------

\documentclass[paper=a4, fontsize=11pt]{scrartcl} % A4 paper and 11pt font size

\usepackage[T1]{fontenc} % Use 8-bit encoding that has 256 glyphs
\usepackage{fourier} % Use the Adobe Utopia font for the document - comment this line to return to the LaTeX default
\usepackage[english]{babel} % English language/hyphenation
\usepackage{amsmath,amsfonts,amsthm,amssymb} % Math packages
\usepackage[vlined,lined,boxed,commentsnumbered]{algorithm2e}

\usepackage{sectsty} % Allows customizing section commands
\allsectionsfont{\centering \normalfont\scshape} % Make all sections centered, the default font and small caps

% --------------------
% Quote Style
% --------------------

\usepackage{tikz}
\usetikzlibrary{backgrounds}
\makeatletter

\tikzset{%
  fancy quotes/.style={
    text width=\fq@width pt,
    align=justify,
    inner sep=.2em,
    anchor=north west,
    minimum width=\textwidth,
  },
  fancy quotes width/.initial={.8\textwidth},
  fancy quotes marks/.style={
    scale=2,
    text=white,
    inner sep=0pt,
  },
  fancy quotes opening/.style={
    fancy quotes marks,
  },
  fancy quotes closing/.style={
    fancy quotes marks,
  },
  fancy quotes background/.style={
    show background rectangle,
    inner frame xsep=0pt,
    background rectangle/.style={
      fill=gray!25,
      rounded corners,
    },
  }
}

\newenvironment{fancyquotes}[1][]{%
  \noindent
  \tikzpicture[fancy quotes background]
  \node[fancy quotes opening,anchor=north west] (fq@ul) at (0,0) {$``$};
  \tikz@scan@one@point\pgfutil@firstofone(fq@ul.east)
  \pgfmathsetmacro{\fq@width}{\textwidth - 2*\pgf@x}
  \node[fancy quotes,#1] (fq@txt) at (fq@ul.north west) \bgroup}
{\egroup;
  \node[overlay,fancy quotes closing,anchor=east] at (fq@txt.south east) {''};
  \endtikzpicture}

\makeatother


% --------------------
% Header and Footer
% --------------------

\usepackage{fancyhdr} % Custom headers and footers
\pagestyle{fancyplain} % Makes all pages in the document conform to the custom headers and footers
\fancyhead[L]{\normalfont \normalsize \textsc{CS673: Graduate Algorithms}} % Class
\fancyhead[R]{\normalfont \normalsize \textsc{Wanzhang Sheng}} % Author
\fancyfoot[L]{} % Empty left footer
\fancyfoot[C]{} % Empty center footer
\fancyfoot[R]{\thepage} % Page numbering for right footer
\renewcommand{\headrulewidth}{0pt} % Remove header underlines
\renewcommand{\footrulewidth}{0pt} % Remove footer underlines
\setlength{\headheight}{13.6pt} % Customize the height of the header

\numberwithin{equation}{section} % Number equations within sections (i.e. 1.1, 1.2, 2.1, 2.2 instead of 1, 2, 3, 4)
\numberwithin{figure}{section} % Number figures within sections (i.e. 1.1, 1.2, 2.1, 2.2 instead of 1, 2, 3, 4)
\numberwithin{table}{section} % Number tables within sections (i.e. 1.1, 1.2, 2.1, 2.2 instead of 1, 2, 3, 4)

% \setlength\parindent{0pt} % Removes all indentation from paragraphs - comment this line for an assignment with lots of text
\setlength{\parskip}{\baselineskip}%
\setlength{\parindent}{0pt}%

% -----------------------------------------------------------------------------
% TITLE SECTION
% -----------------------------------------------------------------------------

\newcommand{\horrule}[1]{\rule{\linewidth}{#1}} % Create horizontal rule command with 1 argument of height

\title{
  \horrule{0.5pt} \\[0.4cm] % Thin top horizontal rule
  \huge Assignment 03 \\ % The assignment title
  \horrule{2pt} \\[0.5cm] % Thick bottom horizontal rule
}

\author{Wanzhang Sheng} % Your name

\date{\normalsize\today} % Today's date or a custom date

\begin{document}

\maketitle % Print the title

% -----------------------------------------------------------------------------
% PROBLEM 1
% -----------------------------------------------------------------------------
\section{}

\begin{fancyquotes}
  (8 points) Exercise $5.1-2$
  Describe an implementation of the procedure $RANDOM(a,b)$ that only
  makes calls to $RANDOM(0,1)$. What is the expected running time of
  your procedure, as a function of a and b? (That is, your function
  will need to call $RANDOM(0,1)$ multiple times --- and the number of
  calls to $RANDOM(0,1)$ will depend upon a, b, and the results of the
  calls to $RANDOM(0,1)$.) It is acceptable for the worst case running
  time of your algorithm to be infinite (though the expected running
  time needs to be finite!)
\end{fancyquotes}

Take the range $[a,b]$ as a segment, and seperate it into two parts.
We keep generating a random number between $0,1$.
If it's $0$, we choose the left part;
If it's $1$, we choose the right part.
Keep this process, until there is only on number in the range, and
return it.

A problem may happend is that when we seperate the range, the one part
may contain one more number than the other one, due to the range may
not be divisible by two.
To solve this problem, we can easily enlarge the range to a speratable
length. If the result is out of range, we just keep generating until
the result is in the range.

So, all the number in the range may be returned by equivelent probility.

The algorithm expected to run under $\Theta(\lg(b-a))$.

\begin{algorithm}[H]
  \SetKwProg{Fn}{Function}{ :}{end}
  \Fn{\textsc{RANDOM}{(a, b)}}{
    \KwData{The upper and lower bound of the range.}
    \KwResult{A random number between the range.}

    \Repeat{}{
      \tcc{Enlarge the range.}
      depth = $\min\{n: 2^n\geq(b-a+1), n \in N\}$\;
      range = $2^{depth}$\;
      min = a\;
      max = a-1+2*range\;

      \tcc{Generate the random number}
      \While{}{
        \tcc{Return if there only one number and in the range.}
        \If{min == max}{
          \If{max <= b} {
            return max;
            \tcc{The function finished.}
          }
          reset the loop\;
        }

        range = range / 2\;
        \eIf{RANDOM(0,1) == 0}{
          max = max - range\;
        }{
          min = min + range\;
        }

      }
    }
  }
  \caption{A random number between the given range}
\end{algorithm}


% -----------------------------------------------------------------------------
% PROBLEM 2
% -----------------------------------------------------------------------------
\section{}

\begin{fancyquotes}
  (8 points) Exercise 5.1-3 Suppose that you want to ouput 0 with
  probability 1/2 and 1 with probabilty 1/2. At your disposal is a
  procedure BIASED-RANDOMM, that outputs either 0 or 1. It ouputs 1 with
  some probability p and 0 with probability (1-p). Give an algorithm
  that uses BIASED-RANDOM as a subroutine, and returns an unbiased
  answer, returning 0 with probability 1/2 and 1 with probability
  1/2. What is the expected running time of your algorithm as a function
  of p? HINT: Your algorithm may call BAISED-RANDOM multiple times, and
  the number of calls to BIASED-RANDOM may depend on the output of
  BIASED-RANDOM.

\end{fancyquotes}

Considering a function called BIASED2, which use BIASED-RANDOM twice until the result
is whether $01$ or $10$.
Since P(BIASED-RANDOM twice return 01) = (1-p)*p = P(BIASED-RANDOM
twice return 10) = p*(1-p).
So the function result is 0 with probability 1/2, and 1 with
probability 1/2.
Then we can construct a UNBIASED-RANDOM to return the last number of
BIASED2, which is 0 or 1 with same probability.

\begin{algorithm}[H]
  \SetKwProg{Fn}{Function}{ :}{end}
  \Fn{\textsc{UNBIASED-RANDOM}{()}}{
    \KwResult{A random number of $1$ or $0$ with same probability}

    \Repeat{}{
      \tcc{BIASED-RANDOM return 0 or 1 with probability of (1-p) and p}
      a = BIASED-RANDOM()\;
      b = BIASED-RANDOM()\;
      \If{a != b}{
        return b\;
      }
    }
  }
  \caption{UNBIASED-RANDOM based on BIASED-RANDOM}
\end{algorithm}


% -----------------------------------------------------------------------------
% PROBLEM 3
% -----------------------------------------------------------------------------
\section{}

\begin{fancyquotes}
  (4 points) Exercise 5.2-1. In HireAssistant, assuming that the
  candidates are presented in random order, what is the probability
  that you hire exactly twice? HINT: Think of all of the cases where
  you hire exactly twice. Be sure you don't miss any!
\end{fancyquotes}

The first one and the best one will always be hired.
We can assume that the best one is the i\textsubscript{th}.
If we hire exactly twice, the best in $[1, i-1]$ should be the first one.
So we have:

\begin{align}
  \begin{split}
    P &= \sum_{i=2}^{n}{\frac{1}{n-1}\times\frac{1}{i-1}}\\
    &=\frac{1}{n-1}\sum_{i=1}^{n-1}{\frac{1}{i}}
  \end{split}
\end{align}


% -----------------------------------------------------------------------------
% PROBLEM 4
% -----------------------------------------------------------------------------
\section{}

\begin{fancyquotes}
  You are tossing balls into b bins. Each toss is independent, and
  each ball is equally likely to fall in any bin. You are going to
  keep tossing balls until there are two balls in one of the bins, at
  which point you stop tossing balls.
\end{fancyquotes}

\begin{enumerate}
\item
  \begin{fancyquotes}
    (3 points) What is the probability that you can make n tosses
    without having 2 balls in any one bin (as a function of n and b)?
  \end{fancyquotes}

  The probability of one toss without conflict is $\frac{b}{b}=1$.

  The probability of two tosses without conflict is
  $\frac{b}{b}\times\frac{b-1}{b}$.

  The probability of three tosses without conflict is
  $\frac{b}{b}\times\frac{b-1}{b}\times\frac{b-2}{b}$.

  $\vdots$

  The probability of n tosses without conflict is $\frac{b!}{b^n\times (b-n)!}$.

\item
  \begin{fancyquotes}
    (3 points) What is the probability that you will stop after
    exactly k tosses (as a function of k and b)?
  \end{fancyquotes}

  The probability of k-1 tosses without conflict is
  $\frac{b!}{b^{k-1}\times (b-k+1)!}$.

  The probability of k\textsubscript{th} toss with conflict is $\frac{k-1}{b}$.

  So the probability of stop after k tosses is
  $\frac{b!}{b^{k-1}\times
    (b-k+1)!}\times\frac{k-1}{b}=\frac{(b-1)!\times(k-1)}{b^{k-1}\times (b-k+1)!}$.

\item
  \begin{fancyquotes}
    (3 points) What is the expected number of tosses before you get a
    bin with two balls, and stop? It is OK if you give a fairly nasty
    formula for this. You don't need to worry about simplifying it.
  \end{fancyquotes}

  \begin{align}
    \begin{split}
      E[X] &= \sum_{k=1}^{\infty}{P[X_k]\times k}\\
      &= \sum_{k=1}^{\infty}{\frac{(b-1)!\times(k-1)}{b^{k-1}\times
          (b-k+1)!}\times k}
    \end{split}
  \end{align}

\end{enumerate}


\end{document}
